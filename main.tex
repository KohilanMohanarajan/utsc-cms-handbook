\documentclass[11pt]{article}
\usepackage{amsmath, amsfonts, amsthm, amssymb, fancyhdr, graphicx, geometry} 
\usepackage{wasysym}
\usepackage{hyperref}
\hypersetup{
    colorlinks=true,
    linkcolor=black,
    filecolor=blue,      
    urlcolor=blue,
}
\geometry{margin=1.5in}
\pagestyle{fancy}
\setlength{\parindent}{10ex}
\title{Student's Handbook to UTSC: CS Year One}
\author{Originally created Kohilan Mohanarajan \\ \\ Updated for 2021 by Will Song and Moe Ali}
\lhead{Student's Handbook to UTSC}
\rhead{CS Year One}
\begin{document}
\maketitle
\newpage
\tableofcontents
\newpage
\section{Foreword}

Congratulations on making it to UTSC! This is a brief guide on making your Program of Study (POSt) in Computer Science at UTSC and some useful bits of information you may want to know before starting your first year!\par
This guide is intended for students who are interested in studying Computer Science. It is written by students so everything here is experience from a student. So as a disclaimer: this is in no way written or supported by the University of Toronto. It is just a student, writing a handbook, for other students. \par
Lastly, this guide is an updated version of \textit{The Computer Science Student's handbook to First Year at UTSC} by Kohlian Mohanarajan. A huge thank you to him as he was the one who helped me understand much of this information. As he puts it "Perhaps the biggest fear a first-year computer science student faces entering University is the possibility of being unprepared." Hopefully, this handbook will cover all your unanswered questions. Any further questions can often be answered (to a significantly better degree) by the staff on campus.\par
Once again, congratulations on making it to UTSC! I can not promise the stay will be fun, or rewarding, or worth your time. Hopefully, this guide just makes your stay here a bit less shit.

\section{FAQs}
\subsection{What is this POSt I keep hearing about?} POSt is your Program of Study. You apply for it after/while completing the 6 core courses. You will need to get above a 3.0 average across the 6 core courses to be guaranteed admission. If you are a tiny bit below, you will be reviewed on a case-by-case basis against your peers based upon how well you performed in certain classes. You apply for POSt typically around the end of first year as you will need to make POSt to enroll in most CS courses past first year. However, if you do not make CS POSt, you can apply to transfer into it later.\par
\subsection{What if I was not accepted for CS?} Luckily for you, you still have an (almost) equal chance to get into CS!
\subsection{What if I was not accepted to UTSC for CS?} You can still apply for CS classes and CS POSt however, students accepted for the CS program will have priority. There is no difference between what you were accepted for beyond that.\par
\subsection{I want to go to UTSG/UTM. What do I do?} You may want to go to a different campus for whatever reason, which I won't debate here. For UTSG focus on CSCA67 and CSCA48. These are equivalent to UTSG's CSC165 and CSC148 respectively. You will need above an 85 average across those two for UTSG based upon the 2017 cutoffs. For UTM there are two requirements, the required minimum CGPA has been set to 2.8. For the 2018-19 requirements, the minimum required marks in MAT102 and CSC148 have both been set to 73$\%$. Apply for an internal transfer early for the best chance. You do not need to pay \$12 to send them your transcript, but if you want absolute assurance you can. If you want you can go to UTSG, you can go to the Gerstein building and talk to the admissions staff (they're nice) or professors if you have questions about their campus. UTSG professors aren't very good at emailing back, the only first year professor I heard from was David Liu, but if you want more information about the campus he is great.\par


\section{Your Backpack}
You will need some things for university. Here is what every computer scientist should consider carrying on them at all times (no electric skateboards, sorry).

\subsection{Your Laptop}
An all too common question is "What laptop should I get?" Please buy a laptop in your price range. For those thinking of buying a sick laptop with a 1070 in it, my personal advice is to first gauge how busy you will be in University, then buy your gaming rig. Trust me as someone who built their PC then had to leave it collecting dust while I worked on assignments.\par
The honest answer is anything will do for your first year. My advice would be to find a light laptop with a comfortable keyboard to use and buy a half-decent mouse. MacBooks can be nice for the UNIX environment but the price tag is very hard to justify.\par
Personally, I use a 13-inch Early-2011 (yes, it is over 7 years old) Macbook Pro with a 2.3 GHz Intel Core i5 processor, 4GB of 1333 MHz DDR3 RAM, Intel integrated graphics, a 300 GB Harddrive, and a 1TB external storage drive. While it is incredibly slow on start-up, and I can't have more than 15 tabs open on Firefox nor run more than 3 applications at once, I still am able to do all my work on it.
\subsection{Your Software}
For first year UTSC has been using Wing as your IDE and Python as your language. If you have something else that you are more familiar with (I just pop open a text editor like Sublime), you will be fine. You may also dabble with $\LaTeX$ on sharelatex.com (and possibly write a CS guide for future students).\par
Beyond that, you will need nothing for class but an internet browser. 
\subsection{Textbooks}
Don't waste money buying them. There is a link at the bottom with all the textbooks you will need for first year. You may encounter some special UTSC only chapter for your MATA31/MATA37 class but the professors are kind enough to upload a scan of the relevant parts online. 
\subsection{Other Things}
\begin{itemize}
    \item T-Card Plus: This is your student ID and a sort of debit card for your T-Bucks (money only usable on campus such as the bookstore, laundry machines on residence, and food) and meal plan (only for food on campus, but you don't have to pay tax on food).
    \item Pencils and Pens: As a CS student you will not have to fill out scantrons. I suggest doing your exams in pencil so you can erase mistakes (professors have their own anti-cheat methods such as scanning the exams before returning it to you so don't get any ideas). No professor I've encountered thus far had a strict 'only-pen can be graded' policy.  
\end{itemize}

\section{The Campus}

UTSC is a relatively small campus, so getting around is fairly easy. It will take you less than a month to be familiar with all the locations listed:
\subsection{Lecture Halls}
\begin{itemize}
\item IC130: Located in the Instructional Centre Building, once you enter the building from the Ellesmere \& Military Trail intersection, turn left on the first floor. It is also accessible from the basement floor if need be.
\item HW216: Located in the Humanities Wing somewhat below the Meeting Place (poorly named, but you will get used to it). To get to head to the staircase (not the large one in the Meeting Place) right behind Copykats.
\item AA112: Located in the Arts and Administration Building. The room is on the side closest to the Student Centre. The entrances to the room from the first floor are up the small staircase to the right of the elevator, and to the right of the door closest to INS. There is an additional way to access it on the first floor but that is mainly intended as an exit so I will leave you to discover it by leaving AA112 from the left door at the front of the room. There are also additional ways to access it from the upper floors however, I have yet to see anyone actually use them. (Who would want to walk up more stairs than they have to!)
\item SY110: Located in the Science Research Wing. It is the only big lecture hall there, you'll easily find it if you go to the main lobby.
\item SW309 \& SW319: Located in the Science Wing, go to the Meeting Place and head up the large stairs, then keep going forward. The two rooms are adjacent to each other, SW309 coming earlier than SW319 if you are coming from the meeting place. They are also accessible from the second floor, albeit with different room names on the doors so it may be a tad bit confusing to find them.
\item HL001: This is part of the newest building on campus: Highland Hall. It has yet to be opened so I have no clue where the room is.
\item AC223: Located in the Academic Resource Centre. While you will not have a lecture here as far as CS cares, it's one of the biggest lecture halls at UTSC, if not the biggest, so expect any large elective classes to be held here.
\end{itemize}

\subsection{Points of Interest}
There isn't much of interest on campus, but beyond the buildings, here are a few things you may want to know about:
\begin{itemize}
    \item \textbf{IC's 3rd and 4th floors:} this is where the CMS Professor's offices are located. 
    \item \textbf{Toronto Pan Am Sports Centre (TPASC):} The University's gym. It has pretty much everything you need and it is where Olympians train so you know it's good. While farther away from campus, it is worth utilizing if you have the time.
    \item \textbf{The Meeting Place:} A large room connecting HW and SW. You'll learn to deal with the name. Most events such as job fairs and club bake sales will be held here, sometimes while passing by you may stumble upon a science fair or conference. 
    \item \textbf{Residences:} There are 3 places that count as residences at UTSC: North Residences, South Residences, and Joan Foley Hall. I will write more about these at a later date. It is your college dorms however, a very small subset of people live on residence. I might create a guide for them later, but definitely worth living in for at least one semester first year if you have the funds.
    \item \textbf{IC308, The AMACSS Office:} If you're looking for some help with your Academics, this is where you'll be able to find your local AMACSS exec, who'll be able to help you. There'll be more on AMACSS later on in the handbook. This is ran by students.
    \item \textbf{Math and Statistics Learning Centre (MSLC):} From their website: The Math and Statistics Learning Centre (MSLC) provides free seminars, workshops, virtual tutoring, individual appointments, and small-group consultations to improve students’ proficiency in various subjects of mathematics and statistics. This is ran by the university.
\end{itemize}

\subsection{Food}
UTSC doesn't really have much in terms of food, to be frank, but what it does have is... it's alright, I guess. While there is a website that lists all the food places, don't be fooled! Many on the list no longer exist. This guide has a comprehensive list of the places that are available. Note that if you do purchase a meal plan, all these vendors will accept it, except for the things in the plaza nearby the TPASC and Nasir's.\par
\subsubsection{The Student Centre}
The Student Centre houses five food stalls and a restaurant underneath, which offer another range of options for those who don't want to shell money out at the Marketplace.
\begin{itemize}
    \item There's KFC Express, which offers a stripped down KFC menu (which ironically doesn't have Chicken drumsticks or thighs).  The favorite pick is the snack box, which gives a pretty good bang for your buck for when you're racing between classes.  Wait times here are actually relatively good, ranging between 3-5 minutes.  To date though, you may find that the fountain drinks are a bit flat.
    \item Hero Burger hosts more options akin to its regular chain restaurant, but you'll find yourself waiting for a long while to get your food.  The drinks here are fizzier than KFC, though. \par
    \item Subway also offers the usual selection, though promotions aren't carried, in case you were looking for that limited-offer habanero sauce sub.  Lines are relatively quick, like your average Subway restaurant.\par
    \item Perhaps the newest of the five restaurants, Treats has a pretty good selection which'll run you between five to ten dollars, so if you're craving a pita wrap fix, go here.\par
    \item Finally for the food stalls, there's Asian Gourmet.  This place fluctuates in quality very often, and will run you a pretty penny too.  Food tends to range between average to dry and bland, but if you're in desperate need of some General Tso's Chicken, then this'll have to be your choice. Note: They have made the local news for not the best reasons. \par
    \item Underneath all this is the Rex's Den. It is your standard diner/restauraunt where you can order to go at the bar (yes there is a bar on campus!) or sit down and eat in one of their booths/couches. The food is often fairly bland and overpriced, but sometimes you can get a good value here. The best value would probably be the chicken breast & drumstick with french fries and (a very pitiful) coleslaw. The best tasting dish would be the mac & cheese although it is overpriced. \par
    \item Outside of the Student Centre, you'll find Nasir Al-Huttam with his hot dog stand.  This food stand is a \textbf{MUST} if you're studying at UTSC.  You have not truly gone to UTSC unless you've had food from Nasir's food stand.  He has everything from veggie dogs to Italian Sausages, fries, poutine and more!  Top notch quality that'll run you, at max, 7\$. \par
\end{itemize}
\subsubsection{Not The Student Centre}
\begin{itemize}
    \item There's a Tim Horton's on the first floor of the Bladen Wing, which is where you'll find yourself before your morning lectures standing in line for 20 or so minutes while the line, which extends out of the actual restaurant, moves at a snail's pace. There's a smaller Kiosk offering more barebones offerings (Tea, Coffee, donuts, muffins) which will take 5 minutes for your morning coffee.
    \item Nearby in the Meeting Place there is a Starbucks which will either have a long line, or a short line. It is your standard Starbucks.
    \item The Marketplace can be found in the hallway leading to the Humanities Wing. It would be the closest thing we have to a university dining hall and consists of small shops like a Pizza Pizza kiosk, a Booster Juice, a Spring Roll, as well as a couple of other food options (sushi, pasta, Indian, lunchroom food, burritos, tacos in a bag, cookies, non-alcoholic drinks). Food here usually will run you between five to ten dollars, and offers a decent variety of options. On Wednesdays there are weekly specials (food they don't offer otherwise).
    \item All the way over at the Instructional Centre (IC) there is La Prep. It will be on the high-roller, side costing around \$10, but they have a great selection. It is a great place to grab a bite to eat while rushing to class or right before exams. Their chocolate cake is good for those of you with a sweet tooth. \par
    \item And to round up the food on campus, there is the new cafe in the new Highland Hall! The cafe is still being named and the hall has yet to be completed, but be excited that a new cafe is opening up! \par
    \item Within the TPASC there are a few food vendors: another Tim Horton's, Booster Juice, Pizza Pizza, and Poolside Bar and Grill.
    \item The off-campus food offerings:  If you're dead tired of the food at school, unfortunately, there isn't much outside, unless you're willing to make a trek.  In a plaza next to the Toronto Pan-Am Sports Centre (TPASC), you'll find a Popeye's, a Reginos, Panam's Joint, the local bar, and Osmow's.  Osmow's is probably the best of the bunch, giving you the best bang for your buck.  If you're daring enough, try the Suicide Shawarma.
    
\end{itemize}

\subsection{The Web}
\begin{itemize}

\item Quercus: located at \url{q.utoronto.ca}. This will be your homebase for most of your math courses, as well as electives. On it will be class material from the professor such as assignments, notes, and maybe even discussion boards. Note that the previous system we have used is called Blackboard (see below) which served much the same purpose.
\item Blackboard: located at \url{portal.utoronto.ca}. While this should be obsolete for you, if a professor mentions something will "be on blackboard" they probably mean Quercus and are just using old lingo.
\item Other Sites: Computer Science courses rarely use Blackboard, instead going for their own website, usually located at a website like www.utsc.utoronto.ca/~instructorname/coursecode. This is where class materials for those courses will be uplaoded, as well as announcements.
\item ACORN: Another huge site you'll find yourself logging into often, this is where you'll be choosing your courses and programs, paying for said courses and programs, and receive your marks for your courses.
\item Piazza: This is the class forum used by all of your CS courses. This is usually where Instructors make their announcements, but the main point of Piazza is to offer a person-to-person helpline for exercises and assignments, be it instructors helping students, or students helping students. Piazza experiences usually range from very helpful to downright mind-bogglingly inane. For those of you interested, if you wish to be a TA activity on Piazza is something professors take into consideration. Have fun! \textbf{A Tale of Caution:} The professors can see who you are even if you post anonymously. So don't ask for help cheating on the forum. \textbf{Another Tale of Caution:} If you are being an asshole to the TA, you may lose privileges such as a pre-run (which is super helpful for getting as many marks as possible!).
\item Markus: This will be where you submit your exercises and assignments for your CS courses. It's a simple process, which requires you to upload the necessary .py or .pdf files needed. Keep in mind to name your files properly, it might end up biting you back if you don't.
 
\end{itemize}

\section{Mandatory Courses}
Like every program, Computer Science isn't without its mandatory courses.  These following six courses will be taken by each and every First Year student, and end up contributing towards your application towards POSt (Which we'll talk about later on in the handbook) Textbooks for the courses that have them (The math courses, basically) will be linked to in the "Useful Links" section of this handbook.\par  \textbf{NOTE:} The content of these courses change from year to year, so there may be some slight discrepancies here and there based on who's teaching or what content's been moved around.

\subsection{Your Professors!}
These are the professors you will be learning from according to the current course timetable if you follow CSCA08, CSCA67, MATA31 in the Fall semester, and CSCA48, MATA22, and MATA37 in the Winter semester. I highly suggest you try to see which professor you learn from the best, and to see their reviews on ratemyprofessor.com.
\begin{itemize}
    \item Anya Tafliovich (CSCA08): Anya has not taught this course before.
    \item Kaveh Mahdaviani (CSCA08): I have no clue who this is. I think they are a new hire.
    \item Anna Bretscher (CSCA67): Anna is your only option for CSCA67 and taught (half of) this course last year. From what I remember she has fairly messy hand writing.
    \item Natalia Bruess (MATA31): Natalia is your only option for MATA31 and taught this course last year. 
    \item Francisco Estrada (CSCA48): Estrada has not taught this course before.
    \item Marzieh Ahmadzadeh (CSCA48): Marzieh has taught this course often over the last year.
    \item Mohammad Najafi Ivaki (MATA22): He can lose himself in the question, but remains very adament to helping students.
    \item Sophie Chrysostomou (MATA22): She has historically taught this course.
    \item Kathleen Smith (MATA37): She has historically taught this course. She is a very dedicated and amazing professor, who even won an award from the student publication.
    \item Richard Pancer: He is the program supervisor, so he will approve all POSt applications. If you have any specific questions about POSt, ask him as a last resort because he can only answer questions specifically pertaining to the Computer Science POSt.
\end{itemize}

\subsection{CSCA08: Introduction to Computer Science I (or Introduction to Computer Programming)}

\href{https://utsc.calendar.utoronto.ca/course/CSCA08H3}{UTSC Calendar Page}\\

\textbf{Note:} There are currently talks of this being completely dropped from the mandatory courses (and thus, dropped form the POSt requirements). It seems like it is, at the very least, being reworked into a different course name. While content should remain the same, this may no longer be accurate.\par
This is the very first CS course you'll be taking at UTSC, usually in your Fall semester.  This course introduces you to the fundamentals of your average high-level programming language, taught through Python.  You'll learn basic concepts such as iterative statements, loops, functions, and Object Oriented Programming (OOP) in this course.\par  
In previous years, there have been three major assignments, 5-6 quizzes, as well as 10-11 smaller exercises.  The exercises are released at the end of every week, usually expected to be completed by the end of the following week.  The exercises are usually released on the instructor's respective course website, and are submitted through Markus, our proprietary submission website.  Exercise content consists of making functions, or writing a specific bit of code, which gets run through an automarker, which assigns a grade based on how many of the test cases your program passes.\par
Quizzes are usually handed out at random during your tutorial, and will cover the topic discussed in class the week before.  Quizzes are also really short, being only a page long, and don't require you to do \textit{too} much thinking.\par
Assignments are some of the biggest pieces of coursework you'll be doing in this course.  They encapsulate multiple weeks' worth of content, and require you to show your understanding of said concepts through the code you write.  Assignments are not only automarked to check for whether or not it will always run properly, but also marked by a TA for documentation and coding style.  The assignments also grow in difficulty, with the last one being the hardest of the three. \par
There are also two Term tests in place of one midterm examination, mainly to give you more chances to do well, in case you don't do well on one of the two examinations.  These tests consist of questions that deal with tracing through given code, writing code given a set of instructions, and Code Mangler questions. (Oh, you'll \textit{love} Code Mangler.)\par

\textbf{Key Advice:} Stay Diligent, get your work done on time, and start the assignments as soon as they're handed out, and you should make it through this course comfortably.


\subsection{CSCA48: Introduction to Computer Science II}

\href{https://utsc.calendar.utoronto.ca/course/CSCA48H3}{UTSC Calendar Page}\\

The second CS course you'll take at UTSC, this course expects you to have a strong understanding of the concepts you learned in CSCA08, as the course content begins to deviate from teaching programming to teaching concepts of computer science, such as Data Structures, Recursion, and Complexity.\par
Like A08, this course has 10 exercises, and 5 quizzes, which take around the same amount of effort as they did in A08, albeit with more complicated material.  There are two big assignments this time around, which serve the same function as they did in A08.\par
The term tests are also more difficult in comparison to A08, so don't let your guard down when writing them.\par

\textbf{Key Advice:} Like before, stay diligent, do your work on time, and in preparation for your Term Tests, work through the supplementary questions given out in Practicals, or try and devise questions of your own.  Questions can come out of the blue in this course.

\subsection{CSCA67: Discrete Mathematics}

\href{https://utsc.calendar.utoronto.ca/course/CSCA67H3}{UTSC Calendar Page}\\

This, unlike the other two CS Courses, is the most unlike Computer Science at first glance, but serves a great importance to your studies moving into subsequent years.  This course has been deemed by the professors as the indicator as to how far one will go with their studies in Computer Science, and with good reason.\par
You may be asking yourself what this course even consists of, and to answer that, we must divide the course into two halves: Proofs, and Counting.\par
In the proofs section of the course, you'll be learning the basics to writing formal proofs.  This will include content such as truth tables, predicates and quantifiers, modulus proofs, and induction.  Induction will be the most important concept from this part of the course, as it carries on to not only second-year courses, but also MATA37, hence why this course is one of MATA37's prerequisites. \par
In the counting section of the course, you'll learn basic combinatorics, such as Combinations, Permutations, and Selections, as well as going over basic Probability, like Bayes Theorem.  This is considered to be the easier portion of the course, but don't take that for granted. \par
CSCA67 consists of 8 exercises, each of which consists of roughly 8-10 questions, given out each week, to be completed by the following week, as well as two big assignments, which contain roughly 7 questions, albeit harder than the exercises.  The assignments are due two to three weeks after they've been handed out.  The exercises, as well as the assignments, are weighted more than the midterm, which makes sense, as the class average for the midterm is always very low.  Most students usually get caught up on the midterm, but don't let that dishearten you.\par
\textbf{Key Advice:} Keep up with the course content, as the exercises for this course are usually due the same day as the exercises for CSCA08.  Practice makes perfect, and strong knowledge of the concepts can serve as a basis for doing well on Assignments and the midterm.

\subsection{MATA31: Calculus I for Mathematical Sciences}

\href{https://utsc.calendar.utoronto.ca/course/MATA31H3}{UTSC Calendar Page}\\

This is the first of the three math courses you'll take in first year, and is your first exposure into theoretical maths.  Based on how they structure the course, it's gonna be 60/40 theory to computation.  You'll learn more about writing formal proofs in this course as well, although not to the same extent as CSCA67. Specifically, \textit{delta-epsilon} proofs are an important part of the course and take up a lot of the midterm.\par
Practice and memorization will help you get along easy in this course.  There are five assignments and five quizzes, each alternating from week to week.  The assignments are long and tedious, but serve as fair practice of the material.  Material-wise, the first six weeks will be new content to many, while the latter six weeks will be familiar content.\par
\textbf{Key Advice:} Practice and memorize, become familiar with the content, as this is the easier of the two calculus courses this year.

\subsection{MATA37: Calculus II for Mathematical Sciences}

\href{https://utsc.calendar.utoronto.ca/course/MATA37H3}{UTSC Calendar Page}\\

This is the second Calculus course you'll take in first year, and by far, the hardest.  This is considered by many to be a GPA killer, and usually ends up being the bane of most people's POSt applications. It is a continuation of Calculus 1, but the vast majority of this course will be new to you if you didn't take AP Calc in High School.\par
This course deals with Reimann Sums, Integration, Sequences, and Series from a very theoretical aspect.  Like A31, there are 5 quizzes and 10 assignments, however this time, the order in which you hand in assignments and do quizzes is randomized, meaning that as you walk into your tutorial each week, you won't know whether you hand in the assignment you did over the past week, or if you're writing a quiz based on the material from the assignment.  This forces you to complete every assignment and learn the material, and learn the material you shall if you want to succeed in this course. \par
The midterm and final exam are comprised of questions that not only resemble those you've seen, but test your knowledge of the material with new question types, and questions you may have never seen prior. There are even questions that look like ones you've seen and practiced, however their solution is far from what you have practiced before! This will likely be your worst class, even to those of you who achieve a 4.0 in every other course. Good luck to you as you go through this course, you will need it.\par
\textbf{Key Advice:} STUDY. HARD. Don't slack off, or fall behind with the course material, as it will only harm you further.  True mastery and understanding of the content is necessary if you want to do well in this course.

\subsection{MATA22: Linear Algebra I for Mathematical Sciences}

\href{https://utsc.calendar.utoronto.ca/course/MATA22H3}{UTSC Calendar Page}\\

Linear Algebra is easier to grasp but harder to execute of the two maths you will be taking in your winter semester. MATA22 is a new course (formerly MATA23) to make it a Mathematical Science-focused course.\par
Going to class will be a chore, as it will be printing out an outline from the website, then filling it in during class like an advanced guided fill-in-the-blank.\par
This is a course where the textbook will actually be very useful; in fact, the textbook is perhaps your best resource for this course, so use it to the best of your ability. In this course, you will go over Cartesian Linear Algebra, Matrices, Determinants, Vector Spaces, Eigenvalues, and Eigenvectors, amongst other concepts. \par
MATA22 had 5 quizzes given in tutorials (each a week apart). The assignments are very useful practice for the quizzes as questions may be ripped directly from them. All your other marks will be determined by a midterm or final.\par
\textbf{Key Advice:} If you want to do well in this course, do the assignments, as questions for the quizzes are ripped from them, and just get a good understanding of the theoretical aspect too.

\section{Course Averages \& CS POSt}
While reading through this handbook, you may have encountered this term a couple of times by now, and you may be left wondering what the heck a POSt even is.  When you're accepted to UTSC, you're accepted into the Computer Science, Mathematics, and Statistics Department.  What this means is that you technically aren't in a Computer Science Major just yet.\par
A subject POSt (\textbf{P}rogram \textbf{O}f \textbf{St}udy) is what you apply to at the end of the year, and is the respective Major/Specialist/Minor that you go into after first year.  For instance, you can choose to be a Computer Science Major after first year, alongside two minors, or do a double major in Computer Science and Management, alongside one minor.  There are also specializations, such as the Software Engineering Stream or the Comprehensive Stream, which are more focused on specific parts of Computer Science. These are usually the Subject POSts most Computer Science students end up applying to.\par
However, unlike most other disciplines, these Subject POSts are limited, which means that certain criteria must be fulfilled in order to apply.  As of April 2018, the current \textit{guarantee} (This is important to note) to get a Subject POSt of your choosing is a 3.00 GPA or above across your six mandatory courses.  This is a guarantee in that if you achieve this GPA, then you are automatically guaranteed a spot in your Subject POSt, no matter what.  With that being said, the 3.00 GPA is not a hard cutoff, in that if you do not achieve that GPA across your core courses, you are not removed from the Computer Science program.  Those who aren't able to achieve the 3.00 GPA are then put into consideration, and from that, a certain amount of people are chosen to enter their Subject POSt.\par
Now be prepared, the next section will be shocking. These are the average grades of the 6 core courses:\par
\begin{center}
    \begin{tabular}{|c|c|c|}
    Class&Grade&GPA \\
     CSCA08&C+&2.3\\
     CSCA48&B-&2.7\\
     CSCA67&C+&2.3\\
     MATA31&B-&2.7\\
     MATA37&C&2.0\\
     MATA22&C-&1.7 
    \end{tabular}
\end{center}
The average GPA of a student taking these courses is 2.28, which is a large cut below the 3.0 needed for the guaranteed admission. Unfortunately, only 30-40\% of students will make it to CS POSt. Stay diligent as this number has been the ugly and unfortunate truth of the program for the past many years. Furthermore, your POSt admission for CS is entirely dependent upon these courses and not electives which can boost your grade.\par
While not achieving that 3.00 does not outright mean that you've been denied from your Subject POSt, it means that your chances of being accepted are at the mercy of the CMS department.  You'll find POSt to be a huge point of contention among first years, as it's considered the ultimate goal for the end of the year. At the end of the day, while getting into your subject POSt is something that everyone should be striving for, don't let it be the end of your career in Computer Science if you don't get in.\par
If you do not make it to CS POSt, but are still determined to do computer science, one consideration is to enroll into a Statistics Specialist: The Statistical Machine Learning and Data Mining Stream. This will give you many of the CS courses, while having a more manageable cutoff (2.5).\par
Another consideration is enrolling into a separate major (such as Math or Stats) then applying to transfer subject POSts in your second year. I don't know how many people make it through this method though.\par
A last (weird) consideration: If you suck at the math courses (MATA31, MATA37, MATA22) but are amazing at the CS courses (CSCA48 and CSCA67), consider submitting an internal application to UTSG or UTM. The earlier the better. You may make POSt there but not at UTSC.\par
Good luck to you.


\section{Electives: Birds, Breadths, and Bamboozles}
As with any other University Program, students are required to take courses that don't pertain to their program, perhaps to broaden the scope of their learning experience. Luckily the Uuniversity of Toronto has very barebones requirements, asking that we only take one course in 5 different categories. Obviously, this would lead to many seeking the easy way out, which is to say, finding the easiest courses that can help them fill their necessary requirements.  The following list of courses shouldn't be taken as the final word, as mentioned above, course structuring can vary from year to year, sometimes even semester to semester.  Do further research before diving into this.

\subsection{Some notes}
\begin{enumerate}
\item Nearly all CS courses cover your Quantitative Reasoning requirement. Later on, you will take CSCD03 (mandatory if you are getting a CS degree) which will cover your Social and Behavioural Sciences requirement.

\item You don't \textit{need} to finish all your breadth courses by first year. If you can't take a course because of scheduling conflicts or whatnot, then don't fret it. You just need to complete them before you graduate. However, I strongly recommend you do take as wide of a variety of breath courses as you can early on as you may find a subject you like better than CS.

\item You cannot take 1 credit (two courses) that satisfy one breadth and replace another breadth.  That is to say, you can't take two Arts, Literature, and Language courses, and have the second one replace your History, Philosophy, and Cultural Studies breadth.

\item You can CR/NCR a breadth requirement course, and have it still count towards your degree. This means if you get a 51 in the course, it will show up as a credit on your transcript AND it will not negatively impact your cumulative GPA. USE THIS!!!

\item On top of your Breadth requirements, you also need to fulfill a writing requirement as well before you graduate.  For your convenience, each course has been labelled as to whether or not it satisfies the writing requirement.

\item Unlike breadth requirements, if you CR/NCR a writing requirement course, it won't count towards your degree.  It will count towards your breadth requirement, but not your writing requirement.
\end{enumerate}

\subsection{Arts, Literature, and Language}
\subsubsection{ENGB02: Effective Writing in the Sciences}

\href{https://utsc.calendar.utoronto.ca/course/engb02h3}{UTSC Calendar Page}\\\\

A very easy course, which consists of light coursework, an essay, a midterm, and a final.  The course exists to teach science students how to write proper.  Worth taking if you're looking to improve your writing skills.  Space is limited for this course, so keep that in mind.\\

\textbf{Bird or Bamboozle?:} Bird for sure.\\

\textbf{Does it satisfy the writing requirement?:} No

\subsubsection{ENGB61: Creative Writing (Fiction)}

\href{https://utsc.calendar.utoronto.ca/course/engb61h3}{UTSC Calendar Page}\\\\

This class has a lot of writing in it, but if your a lover of fiction and story-telling and want to write your own work, you'll end up having a great time. In general, ENGB61 is an introductory course to fiction-writing that goes over the structure of effective storytelling   through prose, character, tone and narrative. You'll end the course with a personal portfolio of stories that you've written and edited. With there being small classes, you'll get a lot of feedback on your work and its easy to make friends. The course marking consists of weekly writing prompts/challenges, three short fiction pieces, and a final portfolio that is a review/edit of two of your short fiction pieces. Namely, there is no midterm or final. \\

\textbf{Bird or Bamboozle?:} Bird, if you enjoy writing and reading stories. Otherwise, bamboozle.\\

\textbf{Does it satisfy the writing requirement?:} No

\subsubsection{VPMA95: Elementary Musicianship}

\href{https://utsc.calendar.utoronto.ca/course/VPMA95H3}{UTSC Calendar Page}\\\\

If you've got any experience whatsoever in music theory and/or have read sheet music before, this course is a breeze.  The course assumes you have no formal knowledge of music theory whatsoever, so you'll be starting off with the basics, like notes on a staff, major/minor scales, intervals, circle of fifths, etc.  The toughest thing you will have to do is memorize a few modes and embarrass yourself for 2 minutes singing in a private room to the professor. Overall a very easy course, and a very interesting course too.\\

\textbf{Bird or Bamboozle?:} Bird for sure.\\

\textbf{Does it satisfy the writing requirement?:} No

\subsubsection{MDSA01: Introduction to Media Studies}
\href{https://utsc.calendar.utoronto.ca/course/MDSA01H3}{UTSC Calendar Page (MDSA01)}\\\\
MDSA01 is one of the "birdiest" courses at UTSC. With three online term tests that are open-book, it's easy to get marks in this course if you try. In fact, most students end up working in groups or discussing answers together for these. Furthermore, with over 600 students taking the course usually (most students prefer taking the class online), course notes and study guides are sent around frequently. There is also a short essay, but it is left open (no guidelines) and with the amount of work you hadn't done for the rest of the course, you definitely won't mind it. The course itself is a societal, economical, political and cultural look at media by dissecting the institutions that produce, control and disseminate it  . To be honest, it can get dry quite easily, but most of the examples, movies and videos shown are entertaining.\\

\textbf{Bird or Bamboozle?:} Bird. For sure.\\

\textbf{Does it satisfy the writing requirement?:} No

\subsubsection{LINA01: Introduction to Linguistics}

\href{https://utsc.calendar.utoronto.ca/course/LINA01H3}{UTSC Calendar Page}\\\\
This is like VPMA93, though some could argue the content is easier than the former.  Again, if you've got great memorization skills, go for this course. \\

\textbf{Bird or Bamboozle?:} This one's a bit on the fence, as you may or may not find it easy.\\

\textbf{Does it satisfy the writing requirement?:} Yes

\subsection{History, Philosophy, and Cultural Studies}
\subsubsection{VPMA93: Listening to Music}

\href{https://utsc.calendar.utoronto.ca/course/VPMA93H3}{UTSC Calendar Page}\\\\
This one is a deceiver.  While the content of this course could be deemed as easy, there is a lot of material to memorize.  To memorize all the material in this course, and still have time to do well in your other courses, you'll need to have good memorization skills.\\

\textbf{Bird or Bamboozle?:} Issa Bamboozle, folks.\\

\textbf{Does it satisfy the writing requirement?:} No

\subsubsection{PHLA10: Reason and Truth, and PHLA11: Introduction to Ethics}
\href{https://utsc.calendar.utoronto.ca/course/PHLA10H3}{UTSC Calendar Page (PHLA10)}\\
\href{https://utsc.calendar.utoronto.ca/course/PHLA11H3}{UTSC Calendar Page (PHLA11)}\\\\
Both of these courses fulfill a writing requirement as well, and are possibly two of the easiest ways to get it.  If you've got an open mind, and can sustain an argument well, then both of these courses will be cakewalks for you.\\\\ 
PHLA10 deals with the common-perceived side of Philosophy, asking the hard-hitting questions about our existence and religion.  PHLA11, on the other hand, talks about the philosophy of ethics, through the eyes of both Kantian Contractualism and Welfare Consequentialism.  They'll really get you thinking too, so the interest factor is also high in these courses.  Definitely take the time to take these courses if you can.\\

\textbf{Bird or Bamboozle?:} Bird as it doesn't require any work when compared to your CS courses. However, the average is still a B- in both these courses.\\

\textbf{Does it satisfy the writing requirement?:} Yes

\subsection{Social and Behavioural Sciences}
\subsubsection{SOCA03: Introduction to Sociology}
\href{https://utsc.calendar.utoronto.ca/course/soca03y3}{UTSC Calendar Page}\\\\
Another case of VPMA93 syndrome, this is a course that requires you to memorize a lot of stuff, but there are other things like essays and journal entries to soften the blow along the way.  Like the Philosophy courses, this course will require you to think outside of the box. Do note that this is the first year that SOCA03 is being offered, taking the place of both SOCA01 and SOCA02.  This means that this course will span an entire year rather than just one semester.\\

\textbf{Bird or Bamboozle?:}  If you can put in the time, this can be a bird course for you.\\

\textbf{Does it satisfy the writing requirement?:} No

\subsubsection{MGTA01/MGTA02: Introduction to Canadian Business}
\href{https://utsc.calendar.utoronto.ca/course/MGTA01H3}{UTSC Calendar Page (MGTA01)}\\
\href{https://utsc.calendar.utoronto.ca/course/MGTA02H3}{UTSC Calendar Page (MGTA02)}\\\\
Considered to be some of the easiest courses at UTSC, these ones are just pure memorization.  A02 is a bit harder than A01, but the material in general is very simple and intuitive to memorize, making it less of a burden than something like VPMA93.\\

\textbf{Bird or Bamboozle?:} Bird.\\

\textbf{Does it satisfy the writing requirement?:} No

\subsubsection{ANTA02: Introduction to Anthropology (Society, Culture and Language)}
\href{https://utsc.calendar.utoronto.ca/course/ANTA02H3}{UTSC Calendar Page}\\\\
ANTA02 is a relatively easy course. It talks about the differences between cultural structure, handling social stratification and societal standards while giving you an insight on cultural traditions that you may have not known about. Almost all the professors teaching it are kind and easy to approach, the midterm consists of simple multiple choice and participation marks are awarded just for attending tutorial. The only problem is the two writing assignments and short essay on the final. However, the marking scheme is usually not strict and this shouldn't be a problem if you don't mind doing some write-up.\\

\textbf{Bird or Bamboozle?:} Bird, if you don't mind a bit of writing.\\

\textbf{Does it satisfy the writing requirement?:} Yes


\subsubsection{CSCD03: Social Impact of Information Technology}
\href{https://utsc.calendar.utoronto.ca/course/CSCD03H3}{UTSC Calendar Page}\\\\
One of the few, if not only, Computer Science courses that fulfill a breadth that isn't Quantitative Reasoning, CSCD03 is a course that all Computer Science students need to take before they graduate.  If you don't want to have to take a Social Science breadth, just wait until fourth year and take this course and satisfy the Social and Behavioural Sciences Breadth.\\

\textbf{Bird or Bamboozle?:} Apparently a bird.\\

\textbf{Does it satisfy the writing requirement?:} No

\subsection{Natural Sciences}
\subsubsection{EESA06: Introduction to Planet Earth}
\href{https://utsc.calendar.utoronto.ca/course/EESA06H3}{UTSC Calendar Page}\\\\
Another case of VPMA93 syndrome, just pure memorization.  Depending on when you take the course, you might also have a poster project worth 40$\%$ of your mark, so that alleviates things.  There's a lot of test banks and resources out there that make this course a bunch easier, so resource-wise, this course is a gold mine.

\textbf{Bird or Bamboozle?:} Bird, but you'll have to memorize a lot.\\

\textbf{Does it satisfy the writing requirement?:} No

\subsubsection{EESA09: Wind}
\href{https://utsc.calendar.utoronto.ca/course/EESA09H3}{UTSC Calendar Page}\\\\
Yes, the course's name really is, "Wind".  Don't take it.  Out of all the courses on this list, this would be the one that I personally would recommend the least.  Actually, if you're willing to spend one C/NCR credit on a course, let it be this one.  This has got to be some of the most mundane, boring material that I've ever had to sit through and study.  To make matters worse, the amount of material to remember is also insane.  There are weekly quizzes on blackboard, but unlike other courses like MGTA01, the questions are all randomized, so it works against your favor.  The lectures, which are 3 hours long, will feel like the longest three hours of your life.

\textbf{Bird or Bamboozle?:} BAMBOOZLE.  If you're a masochist, or want to experience masochism, take this course.  Otherwise, steer clear of it.\\

\textbf{Does it satisfy the writing requirement?:} No

\subsubsection{ANTA01: Introduction to Anthropology: Becoming Human}
\href{https://utsc.calendar.utoronto.ca/course/ANTA01H3}{UTSC Calendar Page}\\\\
Considered to be a very easy course, ANTA01 consists of 10 \% just for showing up to tutorials, and a 1500-2000 word essay that they give  a month to complete.  It will involved a bit of memorization, so do keep that in mind.\\

\textbf{Bird or Bamboozle?:} Bird.\\

\textbf{Does it satisfy the writing requirement?:} Yes

\subsubsection{PHYA10: Physics I for Physical Sciences}
\href{https://utsc.calendar.utoronto.ca/course/PHYA10H3}{UTSC Calendar Page}\\\\
PHYA10 is the kind of course that appears to be easy but is actually the devil in disguise. Being a review of high-school level physics, this course seems like a perfect fit for the Natural Sciences credit. However, weekly online quizzes, weekly online assignments and 3-hour obligatory labs every week (that are also marked!) make PHYA10 a real time-sink when you have far more important work to do for your compulsory courses. Still, the class seems to be a popular Natural Sciences choice among computer science students who took/enjoyed high school Physics, and some even end up appreciating the course. And beside all of this, it is considered to be easier than its counterpart PHYA11 (Physics I for Life Sciences). \\

\textbf{Bird or Bamboozle?:} Bamboozle.\\

\textbf{Does it satisfy the writing requirement?:} No


\subsection{Quantitative Reasoning}
You're in Computer Science. This one is a freebie (your CS Courses satisfy this breath requirement).

\section{Co-op at UTSC}
If you're a computer science student at UTSC, you're more than likely also enrolled in the co-op program.  You may have many questions about the co-op program here, and what the experience consists of, and this paragraph will do its best to give you a brief glimpse into this experience.\par
Before your first work term, you'll need to take two prep courses, and one work term search course.  COPD01 is the first course you take, usually in the fall of your first year if you were admitted into coop computer science, or in the summer, if you opt to start your coop journey in second year.  This course is...somewhat bland, to be quite honest.  It's made to serve as a precursor of sorts to the subsequent courses.  You'll be learning some skills like writing a resume and writing a cover letter.  If it's offered online, I'd recommend taking it online.\par
COPD03 is the next course you take, usually in the winter of your first year, or the summer after.  Which semester you get COPD03 scheduled for is up in the air, and at the discretion of the coop office.  If you get COPD03 scheduled for the winter, that means your first work term will be the fall of that year.  If you get COPD03 scheduled for summer, then your first work term is the winter of the next year.  COPD03 is the most worthwhile co-op prep course, in my opinion.  You'll be going more into depth of what applying for jobs is like.  You'll also engage in mock interviews to help prepare you for your work term search.\par
COPD11 is the final course you'll take, during the semester before your work term.  This is a generally bland course, but the bulk of the work is applying for a job.  This is done through the proprietary CSM portal.  You can also apply for jobs outside of CSM, though they need to meet certain criterion.\par
One thing that you need to keep in mind is that in order to maintain your placement in the coop program at UTSC, you need to maintain a 2.5 cGPA.  

\section{Social Life at UTSC: Yes, it exists.}
One may think that Computer Science is a very Xenophobic discipline, and that's...I mean, I guess it holds true in certain facets, but that's besides the point.  A large part of doing well, but also having a fulfilling experience at UTSC is being involved in the community.  Networking is a large part of Computer Science (The social kind, not the one that uses Sockets), and will be pivotal to even the courses you take.\par
Try and get yourself into a group chat, be it one for individual courses, or the CS program in general.  This is a great way to meet new people, but also get more direct help.  Who knows, someone in that group chat you barely talk in might have the answer to that burning question you've been having about some piece of coursework, and conversely, you, implying you are the benevolent reader I assume you to be, can give help to those who need it.  Together, you can build strong bonds that will carry throughout subsequent years. (Unless of course people drop out.  What? It happens.)\par
Another thing to do is to get involved with organizations, clubs and teams.  These provide a release from your coursework, and can even teach you something new.  Attending Hackathons are also great experiences, and can really help expand your network, while acquiring new skills and swag.  If you do nothing but study, University life is bound to get boring quite quickly.  At the end of the day, University is just as much a social experience as it is a learning experience, so make the best of your time here before it's gone.

\section{Computer Science Extracurriculars}
Albeit being a relatively small campus, there are several clubs, teams, and organizations at UTSC.  Of those clubs, there are currently two main ones that serve the Computer Science Community: AMACSS and CSEC.  While these two are the CS-related clubs, feel free to try out for other teams and clubs as well!

\subsection{AMACSS}
AMACSS, or the Association of Mathematical and Computer Science Students, is UTSC's Main DSA (Departmental Student Association), representing students from the Mathematics, Statistics, and Computer Science Students. They host a variety of events throughout the year, such as mixers, gaming nights, and most importantly, Seminars.\par
They will host review seminars for most, if not all first year Computer Science, Mathematics, and Statistics courses before midterms and finals, which can be used as a resource to help one prepare further. They also have an academics page on their website filled with review slides, worksheets, past finals and other documents to help you study. You can buy a Membership card from them at the beginning of the year for a small fee, which will grant you access to all their non-seminar events.  My recommendation, in terms of their seminars, is to try and attend at least a couple before deciding whether or not their review seminars are helpful to your studying patterns. 
\subsection{CSEC}
CSEC, the Computer Science Enrichment Club, is a relatively new club at UTSC, which serves to further your understanding of computer science, while teaching you concepts and skills you may not learn in your typical classroom experience.\par
CSEC-A, the Algorithms Division, goes over theoretical concepts and skills that will benefit those who attend Coding Competitions, but also serves to teach those who aren't familiar with said concepts as well.  The CSEC-S division serves to teach Network Security, and gets people familiar with CyberSecurity fundamentals.  The CSEC-W teaches Web Development for both newcomers and those who might need brushing up on their knowledge, equipping people with skills that can benefit them in the workforce, and at Hackathons.\par
Each division meets once a week, every week, and hosts occasional seminars dealing with other topics like building a PC and perfecting a resume for Job Applications.

\subsection{UTSC Robotics}
I'm just going to quickly plug a club I am currently creating on creating with some fellow upper year students. This will be our first year and we hope you all come to see what we are going to do! Note that this is going to be largely entry-level stuff so if you have no experience in robotics you'd be perfect to come check it out! If you are an experienced FIRST competitor who can build a robot with both hands tied behind your back, tell us! Depending on interest we can expand to more experienced people.

\subsection{Hackathons}
As mentioned before, Hackathons are a great way to expand your network and your skills.  You may be asking, what exactly \textit{is} a Hackathon?  Simply put, Hackathons are competitions that usually occur over the weekend, either over the span of 24 or 36 hours, in which you and a small team work together to make something, be it an app, a webapp, or a hack using some piece of hardware, that you present and compete with at the end of the time allotment they give you.  Hackathons are a great way to dip your toes into the world of competitive Computer Science, and as mentioned before, a great way to increase your network through the people you meet and the sponsor companies you encounter, but also acquiring lots and lots of free merchandise and t-shirts.\par
If you're willing to lose a bit of sleep and go to your first hackathon, ask around to see if anyone else is going to any upcoming Hackathons, or try and find one of the various Hackathon Facebook groups.  Some notable Hackathons to apply for are Hack The North, HackPrinceton, MHacks, UofTHacks, and UTSC's very own Hack the Valley.\par
One more thing, while Hackathons are fun to go to every now and then, don't let them consume your life.  Your education is what's most important (aside from your health), and it shouldn't have to be marred by the amount of Hackathons you go to.

\section{Conclusion}
In the end, your University experience is yours, and yours alone.  Whether you want to spend it holed up in your room studying, or making memes, no one can truly tell you how to live your life.  Hopefully this handbook has been of some help to you, and the fruits of my boredom provided you with some new insight.  Take a deep breath, and head forward to what awaits you, because it'll be one hell of a ride.

\section{Useful Links}
UTSC's Meme group:\\ \url{www.facebook.com/groups/326488927723078/}\\\\
UTSC Class of 2021 Facebook group:\\ \url{https://www.facebook.com/groups/1870631783212775/}\\\\
A repository containing textbooks you'll need for first year, and more:\\ \url{drive.google.com/drive/folders/0BxrdNKPmoDq-N2s0ekxlY0NWd0k}\\\\
Dates and Deadlines for the 2017/2018 school year.  Useful for figuring out when things are due:\\ \url{hive.utsc.utoronto.ca/public/registrar/Fall%2017-WInter%2018%20dates.pdf}\\\\
The Program Page for Computer Science.  This will tell you what courses you'll need in order to graduate:\\ \url{utsc.calendar.utoronto.ca/specialist-program-computer-science-science}\\\\
An official Timetable planner that you can use to plan courses:\\
\url{https://ttb.utoronto.ca/ttb/#!/}

\section{Making this Handbook Better}
As this handbook was made over the course of an afternoon, it's likely to be riddled with typing errors and formatting errors here and there.  Finding those errors and fixing them will only make this handbook better and more robust for future generations.  If you do find any errors, feel free to email \href{ mailto:kohilan6@hotmail.com}{kohilan6@hotmail.com} with the subject line \textbf{CS Handbook Revision}, and we can work towards getting it fixed.  A plus side of this is you can get your name added to the list of contributors to this handbook, and perhaps may even get the responsibility of updating this handed over to you one day.  Until then, I can only thank you for taking the time to read through this all.
\end{document}
